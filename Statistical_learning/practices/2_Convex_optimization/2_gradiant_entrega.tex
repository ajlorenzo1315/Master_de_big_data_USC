% Options for packages loaded elsewhere
\PassOptionsToPackage{unicode}{hyperref}
\PassOptionsToPackage{hyphens}{url}
%
\documentclass[
]{article}
\usepackage{amsmath,amssymb}
\usepackage{iftex}
\ifPDFTeX
  \usepackage[T1]{fontenc}
  \usepackage[utf8]{inputenc}
  \usepackage{textcomp} % provide euro and other symbols
\else % if luatex or xetex
  \usepackage{unicode-math} % this also loads fontspec
  \defaultfontfeatures{Scale=MatchLowercase}
  \defaultfontfeatures[\rmfamily]{Ligatures=TeX,Scale=1}
\fi
\usepackage{lmodern}
\ifPDFTeX\else
  % xetex/luatex font selection
\fi
% Use upquote if available, for straight quotes in verbatim environments
\IfFileExists{upquote.sty}{\usepackage{upquote}}{}
\IfFileExists{microtype.sty}{% use microtype if available
  \usepackage[]{microtype}
  \UseMicrotypeSet[protrusion]{basicmath} % disable protrusion for tt fonts
}{}
\makeatletter
\@ifundefined{KOMAClassName}{% if non-KOMA class
  \IfFileExists{parskip.sty}{%
    \usepackage{parskip}
  }{% else
    \setlength{\parindent}{0pt}
    \setlength{\parskip}{6pt plus 2pt minus 1pt}}
}{% if KOMA class
  \KOMAoptions{parskip=half}}
\makeatother
\usepackage{xcolor}
\usepackage[margin=1in]{geometry}
\usepackage{color}
\usepackage{fancyvrb}
\newcommand{\VerbBar}{|}
\newcommand{\VERB}{\Verb[commandchars=\\\{\}]}
\DefineVerbatimEnvironment{Highlighting}{Verbatim}{commandchars=\\\{\}}
% Add ',fontsize=\small' for more characters per line
\usepackage{framed}
\definecolor{shadecolor}{RGB}{248,248,248}
\newenvironment{Shaded}{\begin{snugshade}}{\end{snugshade}}
\newcommand{\AlertTok}[1]{\textcolor[rgb]{0.94,0.16,0.16}{#1}}
\newcommand{\AnnotationTok}[1]{\textcolor[rgb]{0.56,0.35,0.01}{\textbf{\textit{#1}}}}
\newcommand{\AttributeTok}[1]{\textcolor[rgb]{0.13,0.29,0.53}{#1}}
\newcommand{\BaseNTok}[1]{\textcolor[rgb]{0.00,0.00,0.81}{#1}}
\newcommand{\BuiltInTok}[1]{#1}
\newcommand{\CharTok}[1]{\textcolor[rgb]{0.31,0.60,0.02}{#1}}
\newcommand{\CommentTok}[1]{\textcolor[rgb]{0.56,0.35,0.01}{\textit{#1}}}
\newcommand{\CommentVarTok}[1]{\textcolor[rgb]{0.56,0.35,0.01}{\textbf{\textit{#1}}}}
\newcommand{\ConstantTok}[1]{\textcolor[rgb]{0.56,0.35,0.01}{#1}}
\newcommand{\ControlFlowTok}[1]{\textcolor[rgb]{0.13,0.29,0.53}{\textbf{#1}}}
\newcommand{\DataTypeTok}[1]{\textcolor[rgb]{0.13,0.29,0.53}{#1}}
\newcommand{\DecValTok}[1]{\textcolor[rgb]{0.00,0.00,0.81}{#1}}
\newcommand{\DocumentationTok}[1]{\textcolor[rgb]{0.56,0.35,0.01}{\textbf{\textit{#1}}}}
\newcommand{\ErrorTok}[1]{\textcolor[rgb]{0.64,0.00,0.00}{\textbf{#1}}}
\newcommand{\ExtensionTok}[1]{#1}
\newcommand{\FloatTok}[1]{\textcolor[rgb]{0.00,0.00,0.81}{#1}}
\newcommand{\FunctionTok}[1]{\textcolor[rgb]{0.13,0.29,0.53}{\textbf{#1}}}
\newcommand{\ImportTok}[1]{#1}
\newcommand{\InformationTok}[1]{\textcolor[rgb]{0.56,0.35,0.01}{\textbf{\textit{#1}}}}
\newcommand{\KeywordTok}[1]{\textcolor[rgb]{0.13,0.29,0.53}{\textbf{#1}}}
\newcommand{\NormalTok}[1]{#1}
\newcommand{\OperatorTok}[1]{\textcolor[rgb]{0.81,0.36,0.00}{\textbf{#1}}}
\newcommand{\OtherTok}[1]{\textcolor[rgb]{0.56,0.35,0.01}{#1}}
\newcommand{\PreprocessorTok}[1]{\textcolor[rgb]{0.56,0.35,0.01}{\textit{#1}}}
\newcommand{\RegionMarkerTok}[1]{#1}
\newcommand{\SpecialCharTok}[1]{\textcolor[rgb]{0.81,0.36,0.00}{\textbf{#1}}}
\newcommand{\SpecialStringTok}[1]{\textcolor[rgb]{0.31,0.60,0.02}{#1}}
\newcommand{\StringTok}[1]{\textcolor[rgb]{0.31,0.60,0.02}{#1}}
\newcommand{\VariableTok}[1]{\textcolor[rgb]{0.00,0.00,0.00}{#1}}
\newcommand{\VerbatimStringTok}[1]{\textcolor[rgb]{0.31,0.60,0.02}{#1}}
\newcommand{\WarningTok}[1]{\textcolor[rgb]{0.56,0.35,0.01}{\textbf{\textit{#1}}}}
\usepackage{graphicx}
\makeatletter
\def\maxwidth{\ifdim\Gin@nat@width>\linewidth\linewidth\else\Gin@nat@width\fi}
\def\maxheight{\ifdim\Gin@nat@height>\textheight\textheight\else\Gin@nat@height\fi}
\makeatother
% Scale images if necessary, so that they will not overflow the page
% margins by default, and it is still possible to overwrite the defaults
% using explicit options in \includegraphics[width, height, ...]{}
\setkeys{Gin}{width=\maxwidth,height=\maxheight,keepaspectratio}
% Set default figure placement to htbp
\makeatletter
\def\fps@figure{htbp}
\makeatother
\setlength{\emergencystretch}{3em} % prevent overfull lines
\providecommand{\tightlist}{%
  \setlength{\itemsep}{0pt}\setlength{\parskip}{0pt}}
\setcounter{secnumdepth}{-\maxdimen} % remove section numbering
\ifLuaTeX
  \usepackage{selnolig}  % disable illegal ligatures
\fi
\IfFileExists{bookmark.sty}{\usepackage{bookmark}}{\usepackage{hyperref}}
\IfFileExists{xurl.sty}{\usepackage{xurl}}{} % add URL line breaks if available
\urlstyle{same}
\hypersetup{
  pdftitle={Método de descenso de gradiente en regresión},
  pdfauthor={Alicia Jiajun Lorenzo, Abraham Trashorras},
  hidelinks,
  pdfcreator={LaTeX via pandoc}}

\title{Método de descenso de gradiente en regresión}
\author{Alicia Jiajun Lorenzo, Abraham Trashorras}
\date{2023-10-16}

\begin{document}
\maketitle

\begin{enumerate}
\def\labelenumi{\arabic{enumi}.}
\setcounter{enumi}{-1}
\item
  Importamos las libreias necesarias
\item
  Simular una muestra de tamaño n = 100 de valores de (xi, yi), i ∈ \{1,
  . . . , n\}.
\end{enumerate}

\begin{enumerate}
\def\labelenumi{\alph{enumi})}
\tightlist
\item
  Para ello, se generan 100 observaciones xi de la uniforme (runif).
\end{enumerate}

\begin{Shaded}
\begin{Highlighting}[]
\NormalTok{n }\OtherTok{\textless{}{-}} \DecValTok{100}
\NormalTok{xi }\OtherTok{\textless{}{-}} \FunctionTok{runif}\NormalTok{(n)  }\CommentTok{\# Genera 100 observaciones xi de la uniforme }
\end{Highlighting}
\end{Shaded}

\begin{enumerate}
\def\labelenumi{\alph{enumi})}
\setcounter{enumi}{1}
\tightlist
\item
  Los errores del modelo, i se generan de la distribución normal de
  media 0 y va- rianza σ2. Por ejemplo, si cogemos σ2 = 1, esto se puede
  hacer con (rnorm(100,0,1)).
\end{enumerate}

\begin{Shaded}
\begin{Highlighting}[]
\NormalTok{sigma }\OtherTok{\textless{}{-}} \DecValTok{1}      \CommentTok{\# Variabilidad de los errores}
\NormalTok{epsilon }\OtherTok{\textless{}{-}} \FunctionTok{rnorm}\NormalTok{(n, }\DecValTok{0}\NormalTok{, sigma)  }\CommentTok{\# Genera errores de la distribución normal}
\end{Highlighting}
\end{Shaded}

\begin{enumerate}
\def\labelenumi{\alph{enumi})}
\setcounter{enumi}{2}
\tightlist
\item
  Los valores de yi se calcularían como yi = β0 + β1xi + i.
\end{enumerate}

\begin{Shaded}
\begin{Highlighting}[]
\NormalTok{beta0}\OtherTok{\textless{}{-}}\DecValTok{5} \CommentTok{\# supongamos que betha0=5 para el modelo}
\NormalTok{beta1}\OtherTok{\textless{}{-}}\DecValTok{3} \CommentTok{\# suponngamos que beta1=3 pare el modelo}
\NormalTok{yi }\OtherTok{\textless{}{-}}\NormalTok{ beta0 }\SpecialCharTok{+}\NormalTok{ beta1 }\SpecialCharTok{*}\NormalTok{ xi }\SpecialCharTok{+}\NormalTok{ epsilon  }\CommentTok{\# Calculamos los valores del  y}
\end{Highlighting}
\end{Shaded}

\textbf{NOTA} Comprobamos el comportamiento para distintas
variabilidades de los errores y betas

\begin{Shaded}
\begin{Highlighting}[]
\CommentTok{\# Crear un vector para almacenar las gráficas}
\NormalTok{stds }\OtherTok{\textless{}{-}} \FunctionTok{c}\NormalTok{(}\FloatTok{0.25}\NormalTok{, }\FloatTok{0.5}\NormalTok{, }\FloatTok{0.75}\NormalTok{, }\DecValTok{1}\NormalTok{)}

\ControlFlowTok{for}\NormalTok{ (i }\ControlFlowTok{in} \DecValTok{1}\SpecialCharTok{:}\FunctionTok{length}\NormalTok{(stds)) \{}
  \CommentTok{\# Definimos sd}
  \CommentTok{\# sd \textless{}{-} stds[i]  \# Comentado para usar directamente stds[i]}
  \CommentTok{\# Error (con desviación típica sd)}
\NormalTok{  epsilon }\OtherTok{\textless{}{-}} \FunctionTok{rnorm}\NormalTok{(}\AttributeTok{n =}\NormalTok{ n, }\AttributeTok{sd =}\NormalTok{ stds[i])}
  \CommentTok{\# Calculamos y}
\NormalTok{  yi }\OtherTok{\textless{}{-}}\NormalTok{ beta0 }\SpecialCharTok{+}\NormalTok{ beta1 }\SpecialCharTok{*}\NormalTok{ xi }\SpecialCharTok{+}\NormalTok{ epsilon}
  \CommentTok{\# Ploteamos y lo almacenamos en la lista}
  \FunctionTok{plot}\NormalTok{(xi, yi, }\AttributeTok{main =} \FunctionTok{paste}\NormalTok{(}\StringTok{\textquotesingle{}Diagrama de dispersión para: nº datos = \textquotesingle{}}\NormalTok{, n, }\StringTok{\textquotesingle{}; sd = \textquotesingle{}}\NormalTok{, stds[i]))}
\NormalTok{\}}
\end{Highlighting}
\end{Shaded}

\includegraphics{2_gradiant_entrega_files/figure-latex/unnamed-chunk-5-1.pdf}
\includegraphics{2_gradiant_entrega_files/figure-latex/unnamed-chunk-5-2.pdf}
\includegraphics{2_gradiant_entrega_files/figure-latex/unnamed-chunk-5-3.pdf}
\includegraphics{2_gradiant_entrega_files/figure-latex/unnamed-chunk-5-4.pdf}

\textbf{Nota} Mostramos la dispersión de los valores de nuestros datos
simulados

\begin{Shaded}
\begin{Highlighting}[]
\FunctionTok{plot}\NormalTok{(xi, yi, }\AttributeTok{main =} \FunctionTok{paste}\NormalTok{(}\StringTok{\textquotesingle{}Diagrama de dispersión para: \textquotesingle{}}\NormalTok{,}\StringTok{\textquotesingle{}nº datos =\textquotesingle{}}\NormalTok{, n, }\StringTok{\textquotesingle{}; sd = 1\textquotesingle{}}\NormalTok{))}
\end{Highlighting}
\end{Shaded}

\includegraphics{2_gradiant_entrega_files/figure-latex/unnamed-chunk-6-1.pdf}

\begin{enumerate}
\def\labelenumi{\arabic{enumi}.}
\setcounter{enumi}{1}
\tightlist
\item
  Escoger la función a minimizar. En este caso, la suma de los residuos
  al cuadrado:
\end{enumerate}

J(β0, β1) =∑(yi − β0 − β1xi)².

\begin{Shaded}
\begin{Highlighting}[]
\NormalTok{suma\_residuos\_cuadrado }\OtherTok{\textless{}{-}} \ControlFlowTok{function}\NormalTok{(beta0, beta1) \{}
  \FunctionTok{sum}\NormalTok{((yi }\SpecialCharTok{{-}}\NormalTok{ beta0 }\SpecialCharTok{{-}}\NormalTok{ beta1 }\SpecialCharTok{*}\NormalTok{ xi)}\SpecialCharTok{\^{}}\DecValTok{2}\NormalTok{)}
\NormalTok{\}}
\end{Highlighting}
\end{Shaded}

\begin{enumerate}
\def\labelenumi{\arabic{enumi}.}
\setcounter{enumi}{2}
\tightlist
\item
  Obtener las derivadas parciales de la función a minimizar. A partir de
  estas, aplicad el algoritmo de método de descenso gradiente (véase los
  apuntes del Tema de Optimización convexa).
\end{enumerate}

En la Sección 2.1 de la práctica de Modelos de regresión lineal con R se
dan los pasos a se- guir para el caso del modelo de regresión lineal
simple

\begin{verbatim}
## No renderer backend detected. gganimate will default to writing frames to separate files
## Consider installing:
## - the `gifski` package for gif output
## - the `av` package for video output
## and restarting the R session
\end{verbatim}

Crea una animación utilizando ggplot2 y gganimate. Puedes utilizar un
gráfico de dispersión para visualizar la convergencia del descenso de
gradiente en cada iteración.

p \textless- ggplot(results, aes(x = beta0, y = beta1)) +
geom\_point(aes(color = J), size = 2) + scale\_color\_gradient(low =
``blue'', high = ``red'') + labs(x = expression(beta{[}0{]}), y =
expression(beta{[}1{]}), title = ``Descenso del Gradiente'') +
transition\_states(iteration, transition\_length = 2, state\_length = 1)
+ enter\_fade() + exit\_fade()

\hypertarget{guardar-la-animaciuxf3n-en-un-archivo}{%
\section{Guardar la animación en un
archivo}\label{guardar-la-animaciuxf3n-en-un-archivo}}

anim\_save(``descenso\_gradiente.gif'', p)

\begin{enumerate}
\def\labelenumi{\arabic{enumi}.}
\setcounter{enumi}{3}
\tightlist
\item
  Como en este caso, tenemos disponible la solución óptima a este
  problema en la función lm. Podemos repetir (en un bucle) varias veces
  los pasos anteriores y hacer una comparativa con los estimadores ( ˆβ0
  y ˆβ1) que devuelve dicha función. Por ejemplo, podría ser de interés
  comparar el valor de la función minimizada, J( ˆβ0, ˆβ1), con nuestros
  estimadores y con los obtenidos con la función lm.
\end{enumerate}

\hypertarget{ajustar-un-modelo-de-regresiuxf3n-lineal-con-lm}{%
\section{Ajustar un modelo de regresión lineal con
lm}\label{ajustar-un-modelo-de-regresiuxf3n-lineal-con-lm}}

lm\_model \textless- lm(yi \textasciitilde{} xi) lm\_beta0 \textless-
coef(lm\_model){[}1{]} lm\_beta1 \textless- coef(lm\_model){[}2{]}
lm\_beta0 lm\_beta1

\end{document}
